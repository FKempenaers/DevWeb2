\documentclass[a4paper, 12pt]{article}
\usepackage[utf8]{inputenc}
\usepackage[T1]{fontenc}
\usepackage[french]{babel}
\usepackage{graphicx}
\usepackage{amsmath}
\usepackage{amssymb}
\usepackage{graphicx}
\usepackage{float}
\usepackage{hyperref}
\usepackage{geometry}

\geometry{hmargin=1.5cm,vmargin=4cm}
\pagestyle{headings}

\title{Rapport final de Développement Web 2}
\author{Lucas Jouvet, Francis Kempenaers et Tao Grolleau}
\date{\today}

\begin{document}

\maketitle

\begin{abstract}
  Dans ce rapport, nous présenterons le travail effectué pour le projet de Développement Web 2 au cours du semestre 6. Nous décrirons l'application et ses principales fonctionnalités, puis nous expliquerons les différentes techologies utlisées, l'origine du code source. Enfin, nous préciserons quelques détails, tels que le temps investi et le rôle de chacun des membres du groupe.
\end{abstract}

\newpage

\section{Description de l'application}

\subsection{Principales fonctionnalités}

\subsection{Différents diagrammes}

\subsubsection{Diagramme UML}

\subsubsection{Templates}

\subsubsection{Schéma Entité-Association}

\section{Technologies utilisées}

\section{Origine du code et sources}

À part la classe RSyntaxTextArea récupérée sur GitHub (\url{http://bobbylight.github.io/RSyntaxTextArea/}), le code est entièrement fait à la main. Évidemment, les sites tels que Openclassroom et StackOverflow ont servi d'aide. 

\section{Temps investi et rôles des membres du groupe}

Sans avoir pris en compte la durée de chaque journée de travail, nous estimerons que nous avons passé tous les trois environ quarantes heures chacun sur le projet.
Les rôles ont peu changé, Francis Kempenaeres a pris en charge le développement du client léger, Lucas Jouvet, celui du client lourd et Tao Grolleau, la mise en place de la base de données. 

\section{Conclusion}

\end{document}
